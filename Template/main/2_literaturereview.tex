%--------------------------------------------------------------------------
% Literature Review
%--------------------------------------------------------------------------

\chapter{Literature Review}
\label{cha: literature-review}

\section{Lorem ipsum}
\label{sec: litrev-lorem-ipsum}

Lorem ipsum dolor sit amet, consectetur adipiscing elit, sed do eiusmod tempor incididunt ut labore et dolore magna aliqua. Ut enim ad minim veniam, quis nostrud exercitation ullamco laboris nisi ut aliquip ex ea commodo consequat. Duis aute irure dolor in reprehenderit in voluptate velit esse cillum dolore eu fugiat nulla pariatur. Excepteur sint occaecat cupidatat non proident, sunt in culpa qui officia deserunt mollit anim id est laborum (Table \ref{tab: intro-table2.1}).\\

\begin{spacing}{1.0}
\begin{longtable}[h!]{ p{3.5cm} p{10.5cm} }

    \caption[Global Forest Resources Assessment categories and definitions.]{Categories and definitions based on the Global Forest Resources Assessment.}
    \label{tab: intro-table2.1}\\
    
    \toprule
    Category & Definition \\
    \midrule
    \endhead

    Forest & Land with tree crown cover (or equivalent stocking level) of more than 10\% and area of more than 0.5 ha. \\[5pt]
    Broadleaved forest & Forest with predominance (more than 75\% of tree crown cover) of trees of broadleaved species.\\[5pt]
    Coniferous forest & Forest with predominance (more than 75\% of tree crown cover) of trees of coniferous species.\\[5pt]
	Bamboo/palm formations & Forest on which more than 75\% of the crown cover consists of tree species other than coniferous or broadleaved species (e.g. tree-form species of the bamboo, palm and fern families).\\[5pt]
	Mixed forest & Forest in which neither coniferous, nor broadleaved, nor palms, bamboos, account for more than 75\% of the tree crown cover.\\[5pt]
	Closed forest ($\geq$ 40\%) & Natural forest where trees in the various storeys and undergrowth cover 40\% of the ground. These formations do not have a continuous dense grass layer. They are either managed or unmanaged forests primary or in an advanced state of reconstitution and may have been logged-over one or more times, having kept their characteristics of forest stands, possibly with modified structure and composition. Typical examples of tropical closed forest formations include tropical rainforest and mangrove forest.\\[5pt]
	Open forest (10 to \textless 40\%) & Formations where trees form a discontinuous layer covering 10 to 40\% of the ground. This forest usually includes a continuous grass layer allowing grazing activities and the spreading of fires.\\[5pt]
	Forest plantation & Forest stands established by planting or/and seeding in the process of afforestation or reforestation. They are either of introduced species (all planted stands), or intensively managed stands of indigenous species, which meet all the following criteria: one or two species at plantation, even age class, regular spacing.\\[5pt]
	
    \bottomrule
\end{longtable}
\end{spacing}

\subsection{Lorem ipsum}

Lorem ipsum dolor sit amet, consectetur adipiscing elit, sed do eiusmod tempor incididunt ut labore et dolore magna aliqua. Ut enim ad minim veniam, quis nostrud exercitation ullamco laboris nisi ut aliquip ex ea commodo consequat. Duis aute irure dolor in reprehenderit in voluptate velit esse cillum dolore eu fugiat nulla pariatur. Excepteur sint occaecat cupidatat non proident, sunt in culpa qui officia deserunt mollit anim id est laborum.

\begin{enumerate}
	\item Lorem ipsum dolor sit amet, consectetur adipiscing elit, sed do eiusmod tempor incididunt ut labore et dolore magna aliqua. Ut enim ad minim veniam, quis nostrud exercitation ullamco laboris nisi ut aliquip ex ea commodo consequat. Duis aute irure dolor in reprehenderit in voluptate velit esse cillum dolore eu fugiat nulla pariatur. Excepteur sint occaecat cupidatat non proident, sunt in culpa qui officia deserunt mollit anim id est laborum;
	\item Lorem ipsum dolor sit amet, consectetur adipiscing elit, sed do eiusmod tempor incididunt ut labore et dolore magna aliqua. Ut enim ad minim veniam, quis nostrud exercitation ullamco laboris nisi ut aliquip ex ea commodo consequat. Duis aute irure dolor in reprehenderit in voluptate velit esse cillum dolore eu fugiat nulla pariatur. Excepteur sint occaecat cupidatat non proident, sunt in culpa qui officia deserunt mollit anim id est laborum.
\end{enumerate}

